\chapter{Ontolog�a de nuestro plan de grado}

\section{Herramienta utilizada:Prot�g�}
Prot�g�\cite{Stanford-Protege:WEB} es un framework para la edici�n de ontolog�as de c�digo abierto. Las ontolog�as creadas con prot�g� se pueden exportar a muy diversos formatos, entre los cuales se incluyen RDF(S), OWL e incluso esquemas XML.
Prot�g� est� construido en Java y admite extensiones creadas por los usuarios por lo que constituye una excelente base para el desarrollo de aplicaciones o prototipos.
Adicionalmente, Prot�g� cuenta con amplica comunidad de usuarios (desarrolladores, docentes, estudiantes e instituciones gubernamentales y corporaciones privadas) que utilizan Prot�g� como base de conocimiento en �reas tan diversas como la biomedicina o la adquisicion de conocimiento.

\section{Explicaci�n de la ontolog�a}
explicar qu� hemos hecho y porqu�.

\section{Instancia UPM (varios ejemplos?)}
aplicar la instancia de nuestro plan. si es posible, incluir varios ejemplos.