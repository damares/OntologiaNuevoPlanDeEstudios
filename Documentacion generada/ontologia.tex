\chapter{Ontolog�a de nuestro plan de grado}

\section{Herramienta utilizada:Prot�g�}
Prot�g�\cite{Stanford-Protege:WEB} es un framework para la edici�n de ontolog�as de c�digo abierto. Las ontolog�as creadas con prot�g� se pueden exportar a muy diversos formatos, entre los cuales se incluyen RDF(S), OWL e incluso esquemas XML.
Prot�g� est� construido en Java y admite extensiones creadas por los usuarios por lo que constituye una excelente base para el desarrollo de aplicaciones o prototipos.
Adicionalmente, Prot�g� cuenta con amplica comunidad de usuarios (desarrolladores, docentes, estudiantes e instituciones gubernamentales y corporaciones privadas) que utilizan Prot�g� como base de conocimiento en �reas tan diversas como la biomedicina o la adquisicion de conocimiento.

\section{Explicaci�n de la ontolog�a}
Para crear la ontolog�a, se ha partido del documento \cite{FIUPM-MemoAneca}. Este documento es el remitido a ANECA\footnote{Agencia Nacional de Evaluaci�n de Calidad y Acreditaci�n - www.aneca.es} para su validaci�n de acuerdo a las normas especificadas en el marco europeo. Aneca, como miembro de ENQA\footnote{European Association for Quality Assurance in Higher Education - www.enqa.eu}, EQAR\footnote{Eurpean Quality Assurance Register for Higher Education - www.eqar.eu} y INQAAHE\footnote{International Network for Quality Assurance Agencies in Higher Education - www.inqaahe.org} es el encargado de validar los planes de estudios y de certificar el cumplimiento de los requisitos marcados por Bolonia

\section{Instancia UPM (varios ejemplos?)}
aplicar la instancia de nuestro plan. si es posible, incluir varios ejemplos.