\chapter{Aplicaciones (autom�ticas)}
Aqui hablaremos de las disitntas aplicaciones, autom�ticas o no, que hemos encontrado y c�mo pensamos que le podemos sacar partido a todo esto.

\section{Mencionar nuestras ideas sobre c�mo sacar partido a esto}

\todo{Hablar los puntos mencionados por }
\begin{itemize}
\item Se supone que la \emph{Description Logic} permite el
  razonamiento sobre la ontolog�a para as� descubrir errores en su
  concepci�n (TBox) o en sus datos (ABox).
\item Presentaci�n de la informaci�n contenida en la ontolog�a para
  diferentes perfiles: profesores, alumnos, secretar�a, legos,
  etc. V�a exportaci�n a html o a herramientas visuales (grafos,
  etc.).
\item La ontolog�a es en s� misma un modelo de datos por lo que se
  puede generar a partir de ella diferentes modelos de datos: UML,
  relacional, no-sql, etc. (Daniel tiene algunas cosas apuntadas de un
  antiguo org).
\end{itemize}

\section{Ejemplo de razonamiento (puede que encuentres una inconsistencia al ir metiendo la info en la ontolog�a,ap�ntalo).}

\section{Ejemplo de vis. de la informaci�n (html).}