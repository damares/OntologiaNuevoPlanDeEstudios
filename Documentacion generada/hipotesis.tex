\subsection{Hip�tesis: herramientas autom�ticas}
Como ya hemos hablado antes, la utilizaci�n de un marco unificado para la representaci�n del conocimiento, nos permite homogeneizar el marco del conocimiento del caso de estudio, de modo que sistema y usuarios comparten la misma visi�n de ese mundo modelizado.
Esta unficaci�n del marco de conocimiento nos es muy �til a la hora de utilizar herramientas que, de forma autom�tica, realicen las m�s variadas tareas sobre el marco creado, como pueden ser la extracci�n de informaci�n o la inferencia.
Este marco �nico se ha modelado utilizando la aplicaci�n Proteg�\cite{Stanford-Protege:WEB}. 
Proteg� es un editor de ontolog�as libre, con una filosof�a "`open source"' que adem�s soporta varios formatos, entre los que destaca OWL\cite{W3C-OWL:WEB}. Proteg� ha sido desarrollado por el centro de investigaci�n inform�tica y biom�dica\cite{Stanford-BMIR:WEB} de la Escuela Universitaria de Medicina de la universidad de  Standford\cite{Stanford-MED:WEB}, con la colaboraci�n de DARPA, eBay, National Cancer Institute, National Institute of Standards and Technology, National Library of Medicine y National Science Foundation, entre otros.
Gracias al formato OWL soportado por Proteg�, a su filosof�a de software libre, y a los apoyos recibidos por parte de diversas instituciones como las arriba mencionadas, existen multitud de herramientas disponibles para trabajar de manera autom�tica con las ontolog�as creadas por Proteg�, ya sea en forma de plug-in's o en forma de herramientas que trabajan directamente sobre la ontolog�a en formato OWL.
Entre la legi�n de herramientas encontradas, destacamos:
\begin{itemize}
\item OWL-Lint\cite{Stanford-OWL_Lint:WEB}. Herramienta para el test autom�tico de ontolog�as para depuraci�n, control de calidad, etc.
\item Outline and Existential Tree Views\cite{Stanford-OETV:WEB}. Herramienta que nos permite la navegaci�n a trav�s de una ontolog�a, no s�lo mediante el uso de la relaci�n existencial es-un.
\item Cloud Views \cite{Stanford-CV:WEB}. Herramienta que permite visualizar la ontolog�a como una nube de tags, cada uno con su correspondiente peso basado en distintos criterios (profundidad, uso, ...).
\item Ontology browser\cite{UniManchester-OB:WEB}. Herramienta que permite la navegaci�n a trav�s de la ontolog�a, y que construye los documentos html de manera din�mica.
\item OWLDoc\cite{Stanford-OWLDoc:WEB}. Herramienta que genera un conjunto de p�ginas html (la mayor�a est�ticas) para su publicaci�n en web.
\item OWL2UML\cite{Stanford-OWL2UML:WEB}. Herramienta que crea un diagrama UML que representa la ontolog�a activa.
\end{itemize}