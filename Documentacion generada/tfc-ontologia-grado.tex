%% Recomendaciones seguidas: http://www.fi.upm.es/?pagina=1475
%% - Paginas: DIN A4, a doble cara
%% - Portada: en breve se publicara un ejemplo
%% - Letra: Times Roman 12 puntos o equivalente en negro
%% - M�rgenes: superior e inferior 3,5 cm, izquierdo y derecho 3 cm.
%% - Superficie del texto: 22,5 cm. de alto (aproximadamente 40
%%   l�neas) y 15 cm. de ancho
%% - Cabecera y pies: fuera de la superficie del texto
%% - Secciones y subsecciones: rese�adas con numeraci�n decimal a
%%   continuaci�n del n�mero del cap�tulo. Ej.: subsecciones 2.3.1.
%% - T�tulos de cap�tulos: en letra may�scula
%% - N�meros de p�gina: siempre centrado en margen inferior, p�gina 1
%%   comienza en cap�tulo 1, todas la secciones precedentes al
%%   cap�tulo 1 en n�mero romano (en min�scula)
%% - Bibliograf�a: seg�n las recomendaciones de la IEEE
%%   (http://www.fi.upm.es/docs/estudios/grado/1475_ieeecitationref.pdf)

\documentclass[twoside,12pt]{book}
\usepackage[
a4paper,
bindingoffset=10mm,
hmargin=30mm,			
vmargin=35mm,
textwidth=150mm,
textheight=225mm,
marginpar=25mm]{geometry}
\linespread{1.1} % Para forzar unas 40 l�neas
\setlength{\parskip}{1.5ex}

%%%%%%%%%%%%%%%%%%%%%%%%%%%%%%%%%%%%%%%%%%%%%%%%%%%%%%%%%%%%%%%%%%%%%%%%
% Codificaci�n de caracteres
\usepackage[latin1]{inputenc}

%%%%%%%%%%%%%%%%%%%%%%%%%%%%%%%%%%%%%%%%%%%%%%%%%%%%%%%%%%%%%%%%%%%%%%%%
% Selecci�n del lenguaje
\usepackage[spanish,english]{babel}

%%%%%%%%%%%%%%%%%%%%%%%%%%%%%%%%%%%%%%%%%%%%%%%%%%%%%%%%%%%%%%%%%%%%%%%%
%% Cabeceras
\usepackage{fancyhdr}

%%%%%%%%%%%%%%%%%%%%%%%%%%%%%%%%%%%%%%%%%%%%%%%%%%%%%%%%%%%%%%%%%%%%%%%%
%% Gr�ficos
\usepackage[final]{graphicx}

%%%%%%%%%%%%%%%%%%%%%%%%%%%%%%%%%%%%%%%%%%%%%%%%%%%%%%%%%%%%%%%%%%%%%%%%
\usepackage{ifdraft}

%%%%%%%%%%%%%%%%%%%%%%%%%%%%%%%%%%%%%%%%%%%%%%%%%%%%%%%%%%%%%%%%%%%%%%%%
%% Tipos de letra
\usepackage{kmath,kerkis}
\usepackage[scaled=0.83]{helvet} %% Helvetica queda demasiado grande
\usepackage[scaled=0.75]{beramono}
\usepackage[T1]{fontenc}

\newcommand{\dedicafont}{\usefont{T1}{phv}{m}{n}\fontsize{14}{17.5}\selectfont}
\newcommand*{\copyrightfont}{\usefont{T1}{phv}{m}{n}\fontsize{6}{7.5}\selectfont}
\newcommand{\sectionalfont}{\usefont{T1}{phv}{b}{n}}
% {64}{80} {56}{70} {48}{60} {32}{40} {24}{30} {18}{22.5} {16}{20} {14}{17.5} {12}{15}
\newcommand{\partnumberfont}{\sectionalfont\fontsize{64}{80}\selectfont}
\newcommand{\parttitlefont}{\sectionalfont\fontsize{48}{60}\selectfont}
\newcommand{\chapternumberfont}{\sectionalfont\fontsize{64}{80}\selectfont}
\newcommand{\chaptertitlefont}{\sectionalfont\fontsize{32}{40}\selectfont}
\newcommand{\sectiontitlefont}{\sectionalfont\fontsize{18}{22.5}\selectfont}
\newcommand{\subsectiontitlefont}{\sectionalfont\fontsize{16}{20}\selectfont}
\newcommand{\subsubsectiontitlefont}{\sectionalfont\fontsize{14}{17.5}\selectfont}
\newcommand{\headertitlefont}{\sectionalfont\fontsize{12}{15}\selectfont}

%%%%%%%%%%%%%%%%%%%%%%%%%%%%%%%%%%%%%%%%%%%%%%%%%%%%%%%%%%%%%%%%%%%%%%%%
%% Tipos de letra en cabeceras de secciones
\usepackage{titlesec}

\titleformat{\part} % command
[display] % shape
{} % format
{\partnumberfont\flushright Part\quad\thepart} % label
{0pt} % sep
{\parttitlefont\flushright} % before
[] % after

\titleformat{\chapter} % command
[display] % shape
{} % format
{\chapternumberfont\flushright\thechapter} % label
{0pt} % sep
{\chaptertitlefont\flushright} % before
[] % after

\titleformat*{\section}{\sectiontitlefont}

\titleformat*{\subsection}{\subsectiontitlefont}

\titleformat*{\subsubsection}{\subsubsectiontitlefont}

%%%%%%%%%%%%%%%%%%%%%%%%%%%%%%%%%%%%%%%%%%%%%%%%%%%%%%%%%%%%%%%%%%%%%%%%
%% Macros para este TFC
\newcommand{\titulo}{Uso de ontolog�as para la implantaci�n del Espacio Europeo de Educaci�n Superior en las titulaciones de grado}
\newcommand{\autor}{Daniel Mart�nez Esteban}
\newcommand{\tutor}{�ngel Herranz Nieva}
\newcommand{\fecha}{Julio 2012}

% Cleardoublepage
\makeatletter
\def\cleardoublepage{\clearpage\if@twoside \ifodd\c@page\else
\hbox{}
\vspace*{\fill}
\begin{center}
% This page intentionally contains only this sentence.
\end{center}
\vspace{\fill}
\thispagestyle{empty}
\newpage
\if@twocolumn\hbox{}\newpage\fi\fi\fi}
\makeatother

\usepackage[textsize=small]{todonotes}

\usepackage{hyperref}

\usepackage{listings}

\usepackage{xcolor}
\definecolor{dkgreen}{rgb}{0,0.6,0}
\definecolor{gray}{rgb}{0.5,0.5,0.5}
\definecolor{light-gray}{rgb}{0.97,0.97,0.97}

%%%%%%%%%%%%%%%%%%%%%%%%%%%%%%%%%%%%%%%%%%%%%%%%%%%%%%%%%%%%%%%%%%%%%%%%
%%%%%%%%%%%%%%%%%%%%%%%%%%%%%%%%%%%%%%%%%%%%%%%%%%%%%%%%%%%%%%%%%%%%%%%%
%% Comienza en libro
\begin{document}

%%%%%%%%%%%%%%%%%%%%%%%%%%%%%%%%%%%%%%%%%%%%%%%%%%%%%%%%%%%%%%%%%%%%%%
%% SELECCI�N DEL IDIOMA
%% (NO TOCAR: S�LO SE ADMITE ESPA�OL)
%% (VER summary.tex para activar otro idioma, ej. ingl�s)
\selectlanguage{spanish}

%%%%%%%%%%%%%%%%%%%%%%%%%%%%%%%%%%%%%%%%%%%%%%%%%%%%%%%%%%%%%%%%%%%%%%%%
%%%%%%%%%%%%%%%%%%%%%%%%%%%%%%%%%%%%%%%%%%%%%%%%%%%%%%%%%%%%%%%%%%%%%%%%
%% Portada, dedicatoria, �ndices, agradecimientos y res�menes
\pagestyle{empty}

\thispagestyle{empty}

\begin{tabular}{cc}
  \begin{minipage}{2cm}
%    \hspace*{-2em}
    \includegraphics[width=2cm]{logos/logofiBN}
  \end{minipage}
  &
  \begin{minipage}{0.75\textwidth}
    \begin{large}
      UNIVERSIDAD POLIT�CNICA DE MADRID\\[1ex]
      FACULTAD DE INFORM�TICA
    \end{large}
  \end{minipage}
\end{tabular}

\vfill

\begin{center}
  \begin{LARGE}
    TRABAJO FIN DE CARRERA
  \end{LARGE}
\end{center}

\vfill

\begin{center}
  \begin{LARGE}
    \titulo
  \end{LARGE}
\end{center}

\vfill

\begin{center}
  \begin{LARGE}
    AUTOR: \textrm{\autor}\\[1ex]
    TUTOR: \textrm{\tutor}\\[2ex]
    \fecha
  \end{LARGE}
\end{center}

%%% Local Variables: 
%%% mode: latex
%%% TeX-master: "tfc-ontologia-grado"
%%% TeX-PDF-mode: t
%%% ispell-local-dictionary: "castellano"
%%% End: 

\cleardoublepage

\hfill
\begin{textit}
  (dedicatoria)
\end{textit}

%%% Local Variables: 
%%% mode: latex
%%% TeX-master: "TFC"
%%% TeX-PDF-mode: t
%%% ispell-local-dictionary: "castellano"
%%% End: 

\cleardoublepage

\pagestyle{plain}
\pagenumbering{roman}

%%%%%%%%%%%%%%%%%%%%%%%%%%%%%%%%%%%%%%%%%%%%%%%%%%%%%%%%%%%%%%%%%%%%%%
%% �NDICE GENERAL
%% (NO TOCAR)
\tableofcontents
\cleardoublepage

%%%%%%%%%%%%%%%%%%%%%%%%%%%%%%%%%%%%%%%%%%%%%%%%%%%%%%%%%%%%%%%%%%%%%%
%% �NDICE DE FIGURAS
%% (COMENTAR SI NO HAY FIGURAS)
\listoffigures
\cleardoublepage

% %%%%%%%%%%%%%%%%%%%%%%%%%%%%%%%%%%%%%%%%%%%%%%%%%%%%%%%%%%%%%%%%%%%%%%
% %% �NDICE DE TABLAS
% %% (COMENTAR SI NO HAY TABLAS)
% \listoftables
% \cleardoublepage

%%% Local Variables: 
%%% mode: latex
%%% TeX-master: "tfc-betfair-ios"
%%% TeX-PDF-mode: t
%%% ispell-local-dictionary: "castellano"
%%% End: 

\cleardoublepage

\chapter*{Agradecimientos}

Agradezco a \ldots

%%% Local Variables: 
%%% mode: latex
%%% TeX-master: "tfc-ontologia-grado"
%%% TeX-PDF-mode: t
%%% ispell-local-dictionary: "castellano"
%%% End: 

\cleardoublepage

\chapter*{Resumen}
\addcontentsline{toc}{chapter}{Resumen}

\todo{Un res�men del TFC}

%%% Local Variables: 
%%% mode: latex
%%% TeX-master: "tfc-ontologia-grado"
%%% TeX-PDF-mode: t
%%% ispell-local-dictionary: "castellano"
%%% End: 

\cleardoublepage

%\begin{otherlanguage}{english}

\chapter*{Summary}
\addcontentsline{toc}{chapter}{Summary}

(Summary here)\todo{traducir el resumen del espa�ol al ingl�s.}

%\end{otherlanguage}

%%% Local Variables: 
%%% mode: latex
%%% TeX-master: "tfc-ontologia-grado"
%%% TeX-PDF-mode: t
%%% ispell-local-dictionary: "castellano"
%%% End: 

\cleardoublepage

%%%%%%%%%%%%%%%%%%%%%%%%%%%%%%%%%%%%%%%%%%%%%%%%%%%%%%%%%%%%%%%%%%%%%%%%
%%%%%%%%%%%%%%%%%%%%%%%%%%%%%%%%%%%%%%%%%%%%%%%%%%%%%%%%%%%%%%%%%%%%%%%%
%% Cap�tulos

%%%%%%%%%%%%%%%%%%%%%%%%%%%%%%%%%%%%%%%%%%%%%%%%%%%%%%%%%%%%%%%%%%%%%%%%
%% Cabeceras
\pagestyle{fancy}
% Remembering the chapter title:
\renewcommand{\chaptermark}[1]{\markboth{\headertitlefont\thechapter\quad#1}{}}
% Remembering section number and title:
\renewcommand{\sectionmark}[1]{\markright{\headertitlefont\thesection\quad#1}}
\fancyhead{} % Clear all header fields
\fancyhead[LE]{\leftmark}
\fancyhead[RO]{\rightmark}
\fancyfoot{} % Clear all footer fields
\fancyfoot[C]{\thepage}
\renewcommand{\headrulewidth}{0pt}
\renewcommand{\footrulewidth}{0pt}
\pagenumbering{arabic}

%%%%%%%%%%%%%%%%%%%%%%%%%%%%%%%%%%%%%%%%%%%%%%%%%%%%%%%%%%%%%%%%%%%%%%
%% CAP�TULOS:
%%  * PONER CADA CAP�TULO EN UN FICHERO
%%  * A�ADIR \cleardoublepage DESPU�S DE CADA input

\chapter{Introducci�n}

	En este cap�tulo haremos un breve repaso hist�rico a trav�s de los diferentes 
	acuerdos y tratados firmados, que han llevado a la educaci�n superior en europa 
	desde un estado fragmentado y sin cohesi�n ninguna, hacia el marco existente 
	en la actualidad - el Espacio Europeo de Educaci�n Superior - que aumenta la 
	compatibilidad y la comparabilidad de los distintos sistemas de educaci�n, 
	respetando siempre su diversidad. 

  \section{Espacio europeo}
    El Espacio Europeo de Educaci�n Superior (EEES en adelante) es un ambicioso
    proyecto puesto en marcha a nivel europeo para armonizar los diferentes sistemas
    universitarios europeos, y dotar de una mayor agilidad a unviersidades y alumnos
    mediante el intercambio de ideas y personas. 

    No tiene como objetivo estandarizar los diversos sistemas de educaci�n superior
    sino aumentar su compatibilidad y comparabilidad. 

    \subsection{Historia}

      \subsubsection{\bfseries \itshape Charta Magna Universitatum}
	En el a�o 1986, la Universidad de Bolonia realiza una propuesta a las m�s
	antiguas universidades europeas de crear una carta magna europea que recoja los
	valores tradicionales de la universidad y que aboge por la difusi�n de sus
	bondades. Esta idea tuvo una gran acogida por las universidades, que durante una
	reuni�n delebrada en Junio de 1987 en la propia Universidad de Bolonia a la que
	asistieron m�s de 80 delegados de diferentes universidades europeas, eligieron
	una comisi�n de ocho miembros encargados de confeccionar la Carta Magna. Esta
	comis��n estaba compuesta por el Presidente de la Conferencia Europea de
	Rectores, los rectores de las universidades de Bolonia, Paris I, Lauven,
	Barcelona, el profesor D.Guiseppe Caputo (Universidad de Bolonia) y el profesor
	D.Manuel Nu�ez Encabo (Presidente de la sub-comisi�n universitaria de la
	Asamblea Parlamentaria del Consejo Europeo. \cite{MCO:WEB}. El documento estaba
	concluido en 1988, y fue ratificado por todos los rectores asistentes a la
	celebraci�n del 
	nonacent�simo aniversario de la fundaci�n de la universidad de bolonia.

	En esa Carta Magna \cite{UNIBOL:MCU-98} se considera que:
	\begin{itemize}
	  \item El porvenir de la humanidad depende del desarrollo cultural, ci�nt�fico y
	  t�cnico y que el epicentro de este desarrollo son las universidades.
	  \item La obligaci�n que la universidad contrae con la humanidad de difundir ese
	  conocimiento para las nuevas generaciones, exige de la sociedad un esfuerzo
	  adicional en la formaci�n de sus ciudadanos.
	  \item La universidad debe ser garante de la educaci�n y formaci�n de las
	  generaciones venideras de modo que �stas contribuyan al equilibrio del entorno
	  natural y de la vida.
	\end{itemize}
	
	Con estos hechos y objetivos, la Carta Magna proclama los cuatro principios
	fundamentales que sustentan la vocaci�n de la universidad. Estos
	principios son:
	\begin{itemize}
	  \item Independencia moral y cient�fica frente al poder pol�tico, econ�mico e
	  ideol�gico.
	  \item Indivisibilidad entre actividad docente y actividad investigadora.
	  \item Libertad de investigaci�n, ense�anza y formaci�n. La universidad es un
	  lugar de encuentro entre personas con la capacidad de transmitir el saber y
	  ampliarlo con los medios puestos a su alcance para la investigaci�n y el
	  desarrollo (profesores) y personas que tienen el derecho, la capacidad y la
	  voluntad de enriquecerse con ello.
	  \item Eliminaci�n de cualquier frontera geogr�fica o pol�tica y fomento del
	  conocimiento intercultural.
	\end{itemize}
	
	Buscando cumplir esos muy ilustres objetivos, finaliza la carta magna
	manifestando la necesidad de alentar la movilidad de profesores y alumnos y de
	establecer una equivalencia, no s�lo en materia de t�tulos, sino tambi�n de
	estatutos, de ex�menes, y de concesi�n de Becas, e insta a los rectores
	firmantes a trabajar para que los Estados y organismos p�blicos implicados
	colaboren en el cumplimiento de las metas acordadas.

	Actualmente, la Charta Magna Universitatum ha sido suscrita por 660
	universidades de 78 pa�ses\footnote{Puede consultarse la lista de universidades
	en http://www.magna-charta.org/magna\_universities.html}. 

      \subsubsection{\bfseries \itshape Declaraci�n de La Sorbona}
	Posteriormente a la ratificaci�n de la Charta Magna Universitatum, el 25 de Mayo
	de 1998 los ministros de educaci�n de Francia, Alemania, Italia y Reino Unido se
	re�nen en La Sorbona (Par�s) y dictan una declaraci�n conjunta que ven�a a dar
	un sustento pol�tico a la declaraci�n recogida en la Charta Magna Universitatum.

	En la Declaraci�n de La Sorbona \cite{UNISOR:DS-98}, realizada en el
	septuacent�simo trig�simo primer aniversario de su fundaci�n, los ministros de
	educaci�n de los pa�ses arriba mencionados vienen a reconocer que Europa es una
	realidad supranacional, con un gran potencial humano gracias a los siglos de
	tradici�n universitaria, y promueve la creaci�n de un marco eurpeo donde las
	entidades nacionales y los intereses comunes puedan relacionarse y reforzarse
	para el beneficio de Europa. Reconoce la p�rdida de movilidad de los estudiantes
	y el empobrecimiento que eso causa a la sociedad, por lo que aboga por una
	vuelta al modelo cl�sico, donde el alumno pueda enriquecerse con estudios
	realizados en otras realidades sociales.

	Establece como sistema de facto el aceptado actualmente (dividido en dos ciclos,
	llamados aqui universitario y de posgrado donde se realizar� una elecci�n entre
	una titulaci�n de master o una de doctorado m�s extensa.) y fija el sistema de
	cr�ditos ECTS como el �ptimo para lograr la comparabilidad y convalidaci�n entre
	diferentes pa�ses. 

      \subsubsection{\bfseries \itshape Declaraci�n de Bolonia}
	El 19 de Junio de 1999 se firma en la Universidad de Bolonia (detonante de todo
	el proceso con su Cartha Magna Universitatum) el documento que da nombre al
	proceso de convergencia hacia un Espacio Europeo de Educaci�n Superior (Proceso
	de Bolonia). 
	La declaraci�n de Bolonia \cite{UNIBOL:DB-99} fu� firmada por 29 ministros con
	competencias en educaci�n superior. Supuso el espaldarazo definitivo a la
	creaci�n del Espacio Europeo de Educaci�n Superior. La Declaraci�n de Bolonia
	fue redactada con la vista puesta en la Charta Magna Universitatum, pero sobre
	todo, en la Declaraci�n de la Sorbona firmada el a�o anterior y que constituye
	el primer espaldarazo pol�tico a la creaci�n de un �rea Europea de Educaci�n
	Superior.

	En ella se insiste de nuevo en la realidad supranacional en que se ha ido
	convirtiendo Europa, y de la concienciaci�n creciente de la sociedad de la
	necesidad de construir un Europa con una s�lida base intelectual, cultural,
	social y cient�fico-tecnol�gica. 

	La declaraci�n de Bolonia fija como meta �ltima para Europa el lograr establecer
	un �rea Europea de Educaci�n Superior, y promocionar el sistema Europeo de
	ense�anza superior en el resto del mundo. Para lograr alcanzar estas metas
	propuestas, se establecen seis objetivos a cumplir a medio plazo (antes de la
	primera d�cada del siglo XXI) que son:
	\begin{itemize}
	  \item Adopci�n de un sistema de titulaciones f�cilmente comprendible y
	  comparable, gracias a la creaci�n del suplemento del diploma, de modo que se
	  facilite la obtenci�n de empleo dentro del marco Europeo y la competitividad de
	  su sistema educativo superior.
	  \item Creaci�n de un sistema basado en dos ciclos, pregrado y grado, de modo que 
	  s�lo se pueda acceder al segundo ciclo una vez se haya superado satisfactoriamente 
	  el primero, con un periodo m�nimo de tres a�os. El diploma obtenido despu�s del 
	  primer ciclo ser� reconocido en el mercado laboral como un nivel adecuado de 
	  preparaci�n. El segundo ciclo conducir� a la obtenci�n del t�tulo de master 
	  o doctorado.
	  % confirmado, en la declaraci�n de bolonia hablan de pregrado y grado para referirse 
	  %a lo que luego ser� grado y master
	  \item Establecimiento de un sistema de cr�ditos para propiciar la movilidad del
	  alumnado. Estos cr�ditos adem�s, se podr�n obtener por la realizaci�n de
	  actividades no lectivas, siempre en el modo en que est� estipulado por la
	  universidad receptora.
	  \item Promoci�n de una movilidad efectiva, venciendo las trabas existentes a la
	  libre circulaci�n, y prestando especial atenci�n a:
	  \begin{itemize}
	    \item el acceso a estudios y otras oportunidades de formaci�n y servicios
	    relacionados de los alumnos.
	    \item reconocimiento y valoraci�n de los periodos de estancia de los profesores,
	    investigadores y personal de administraci�n en instituciones europeas de
	    investigaci�n, ense�anza y formaci�n, todo ello sin prejuicio de sus derechos
	    estatutarios.
	  \end{itemize}
	  \item Incremento de la cooperaci�n Europea para asegurar la calidad de la
	  ense�anza, para lo cual se deber�n desarrollar criterios y metodolog�as
	  comparables.
	  \item Adecuaci�n de las dimensiones Europeas de educaci�n superior que tengan
	  como objetivo el desarrollo curricular, cooperaci�n interinstitucional, mejora
	  de los esquemas de movilidad y de los programas integrados de estudio, formaci�n
	  e investigaci�n.
	\end{itemize}

	Termina la declaraci�n de Bolonia indicando que la creaci�n del �rea Europea de
	Educaci�n Superior se lograr� respetando las singularidades de cada pa�s,
	la diversidad de lenguas, culturas, los diferentes sistemas de educaci�n
	nacional y la autonom�a de las Universidades, y establece un calendario de
	reuniones para realizar un seguimiento del proceso de implantaci�n del �rea
	Europea de Educacion Superior.
	% Confirmado, en Bolonia, a�n se habla de �rea europea de educaci�n superior

    \subsection{El EEES en la actualidad}
    
    Los acuerdos de Bolonia son de una gran complejidad. Hemos de tener en cuenta que en la filosof�a del EEES no est� la homogeneizci�n de la educaci�n superior, sino el establecimiento de un m�todo de comparaci�n y compatibilizaci�n entre los diversos sistemas educativos, respetando la singularidad de cada uno de ellos. 
    El desarrollo de una metodolog�a que permita la comparabilidad y compatibilidad entre los diferentes sistemas educativos permitir� una mayor movilidad de estudiantes, profesores, investigadores y trabajadores, cumpliendo los objetivos pactados en la 
    declaraci�n de Bolonia. Esta mayor integraci�n de los diferentes sistemas educativos
    europeos y la mayor movilidad de las personas que los componen, har�n que Europa 
    de una mano de obra mejor cualificada, lo que sin duda redundar� en una mayor
    competitividad de la industria, ciencia y servicios europeos.
    
    \subsubsection{\bfseries \itshape Bolonia Follow-Up Group}
    
    Desde el a�o 1999 y cada dos a�os, se viene celebrando con regularidad una
      Cumbre Ministerial para hacer balance del progreso realizado en la implantanci�n
      del �rea Europea de Educaci�n Superior, y establecer metas a cumplir de cara a
      la celebraci�n de la pr�xima Cumbre. 

      El encargado de organizar estas cumbres es el BFUG (Bolonia Follow-Up Group). El
      BFUG es el encargado de organizar las cumbres ministeriales y de elaborar el
      plan de trabajo, calendario de seminarios y otras actividades de inter�s para
      todos los participantes en el proceso. El BFUG est� presidido por el pa�s a
      cargo de la Presidencia de turno de la Uni�n Europea y estacompuesto por los
      ministerios afectados por el Proceso Bolonia de los 47 pa�ses integrantes de
      Bolonia, la Comisi�n Europea y varias organizaciones europeas, como la
      Asociaci�n de Universidades Europeas (EUA\footnote{http://www.eua.be/}) o el
      Centro Europeo de la UNESCO para la educaci�n superior
      (UNESCO-CEPES\footnote{http://www.cepes.ro/}), est�s �ltimas como miembros
      �nicamente consultivos.

      Las Cumbres Ministeriales celebradas hasta la fecha han sido Praga
      2001\footnote{http://www.bologna.msmt.cz/PragueSummit/index.html}, Berlin
      2003\footnote{http://www.bologna-berlin2003.de/}, Bergen
      2005\footnote{http://www.bologna-bergen2005.no/}, Londres
      2007\footnote{http://www.bologna-bergen2005.no/} y Lovaina
      2009\footnote{http://www.ond.vlaanderen.be/hogeronderwijs/bologna/}. En esta
      �ltima cumbre se fijaron las prioridades para el �rea Europea de Educaci�n
      Superior, a cumplir en el pr�ximo decenio, que son:
      \begin{itemize}
	\item Acceso equitativo para todos los grupos sociales
	\item Reconocimiento de las habilidades y competencias obtenidas fuera del marco
	puramente acad�mico.
	\item Acceso al mercado laboral.
	\item Ense�anza centrada en el alumno.
	\item Educaci�n, Investigaci�n e Innovaci�n.
	\item Cooperaci�n internacional.
	\item Movilidad del alumnado y profesorado.
	\item Recogida de informaci�n, para un mejor seguimiento de la implantaci�n de
	Bolonia.
	\item Herramientas transparentes para la comparaci�n de titulaciones.
	\item Financiaci�n de las Universidades.
      \end{itemize}
      
      Tambi�n se solicita al BFUG que prepare un plan de actuaci�n para poder avanzar
      en las prioriades marcardas, y se le pide de manera espec�fica que:
      \begin{itemize}
	\item Defina unos indicadores para medir la movilidad de alumnos y profesores.
	\item Considere el modo de que se pueda lograr una mobilidad equilibrada, con un
	flujo total neutro, dentro del �rea Europea de Educaci�n Superior.
	\item Controle el desarrollo de mecanismos de transparencia para ser estudiados
	en la pr�xima Conferencia de Ministros que tendr� lugar en Bucarest en Abril de
	2012.
	\item Cree una red que d� soporte a la expansi�n de Bolonia fuera del �rea
	Europea de Educaci�n Superior, haciendo un uso �ptimo de las estructuras ya en
	funcionamiento.
	\item Siga desarrollando recomendaciones para el an�lisis de los distintos
	planes nacionales para el reconocimiento de cr�ditos.
      \end{itemize}

      Con motivo del aniversario de la Conferencia de Bolonia, en el que se fijaban
      unos objetivos a cumplir antes de 2010, se celebr� el pasado 11 y 12 de Marzo de
      2010 un encuentro entre los pa�ses participantes en el Proceso de Bolonia para
      lanzar definitivamente el �rea Europea de Educaci�n Superior, comprometi�ndose
      todos ellos a cumplir los objetivos marcados en Lovaina 2009 en la fecha
      prevista (2019). Este encuentro viene a plasmar el compromiso firme de los
      pa�ses participantes y las instituciones educativas contenidas en ellos con los
      principios acordados 11 a�os antes en la Universidad de Bolonia.
    
    \subsubsection{\bfseries \itshape Cr�ditos ECTS} 
    El sistema de cr�ditos ECTS es la piedra angular sobre la que se sustenta el EEES, sirve como nexo de uni�n a todos los pa�ses integrados en el proceso de Bolonia, la mayor�a de los cuales han incorporado los cr�ditos ECTS dentro de sus respectivos sistemas educativos.
    %learning outcomes
    Los cr�ditos ECTS son una medida de la carga de trabajo que debe realizar el alumno para obtener las competencias precisas para superar con �xito una asignatura. La carga de trabajo antes mencionada incluye tanto actividades del aula (como clases magistrales, seminarios o ex�menes...) como actividades fuera de ella (realizaci�n de proyectos, estudio solo o en compa��a...).
    Se asignan 60 cr�ditos ECTS a la carga de trabajo de todo un a�o lectivo, con la adquisici�n de las competencias asociadas. De modo general, un a�o lectivo suele constar de ente 1.500 y 1.800 horas de trabajo, lo que nos arroja un valor para el cr�dito de entre 25 y 30 horas de trabajo.
    Los cr�ditos se asignan para la titulaci�n completa y luego se distribuyen proporcionalmente entre los componentes de la titulaci�n en funci�n del peso que cada uno tiene en la titulaci�n. Los cr�ditos se otorgan al alumno al finalizar las actividades formativas marcadas por el programa de estudios y al obtener las compentencias requeridas para la superaci�n de ese hito.
    Es por tanto posible que, si un alumno ya ha adquirido con anterioridad ciertas competencias, los cr�ditos asociados a la adquisici�n de dichas competencias sean convalidados al alumno, una vez que se haya comprobado mediante el oportuno reconocimiento la adquisici�n de dichas competencias.
    Adem�s, los cr�ditos pueden ser transferidos de un programa a otro, ya sea este ofrecido por la misma insitutci�n o por otra diferente. Para que la transferencia de cr�ditos pueda llevarse a cabo, es preciso que la instituci�n receptora de dichos cr�ditos reconozca los cr�ditos otorgados y las competencias reconocidas por la instituci�n oferente. 
    A continuaci�n veremos en detalle las caracter�siticas del ECTS
    \begin{description}
    	\item[El ECTS es un sistema de cr�ditos centrado en el aprendizaje]. Su filosof�a es una ayuda a las instituciones para el cambio de los sistemas de aprendizaje, desde uno centrado en el profesor, donde los cr�ditos antiguos miden horas lectivas, hacia el centrado en el alumno propuesto por Bolonia, donde el cr�dito ECTS mide carga de trabajo para el alumno.
    	Adem�s, los cr�ditos ECTS en conjunci�n con el sistema de adquisici�n de competencias permite establecer una correlaci�n entre la oferta educativa y el mercado laboral, prolongar de manera cont�nua la educaci�n y adquisici�n de competencias al flexibilizar los programas de estudios y facilitar el reconocimiento de los cr�ditos y competencias ya adquiridos, y permite la movilidad entre instituciones de ense�anza, pa�ses y contextos de aprendizaje al facilitar una unidad de equivalencia entre todos ellos. 
    	\item[ECTS y competencias]. Las competencias son el conjunto de aptitudes, actitudes y conocimientos que el alumno posee. 
    \end{description}
    La carga de trabajo  
    
    
    La herramienta que permitir� el cumplimiento de los objetivos fijados en Bolonia, y sobre la cual gira todo el EEES, es el sistema de cr�ditos European Credit Transfer System.
    Los cr�ditos ECTS fueron establecidos por el programa ERASMUS como una herramienta que permit�a gestionar la movilidad acad�mica de estudiantes y profesores acogidos al programa. M�s tarde pasaron a utilizarse en el programa SOCRATES, y finalmente fueron adquiridos como unidad en el EEES, despyu�s de doce a�os de utilizaci�n dentro de los programas antes mencionados.
    Los cr�ditos vigentes antes de la entrada en vigor del EEES eran una medida de las horas lectivas que cada asignatura tiene asignadas. Eran cr�ditos �tiles para medir horas lectivas, pero que de ning�n modo ten�an en cuenta el esfuerzo global que el alumno ten�a que hacer para superar la asignatura. Exist�a por tanto la anomal�a de asignaturas eminentemente pr�cticas, con m�s horas de trabajo fuera del aula que dentro, que ten�an asignados menos cr�ditos que otras asignaturas m�s te�ricas, que requer�an menos trabajo fuera del aula. Es decir, un alumno deb�a dedicar m�s esfuerzo (m�s horas de trabajo) a una asignatura que le reportaba menos cr�ditos, que a otra cuyo aporte de cr�ditos al global de la titulaci�n era mayor y requeria menos horas de trabajo.
    Los cr�ditos ECTS vienen a eliminar la problem�tica sobre la contabilidad de los cr�ditos, unificando la unidad de medida del trabajo preciso para la obtenci�n del t�tulo. Los cr�ditos ECTS contabilizan la carga de trabajo necesaria para el alumno para la superacion de una materia.
    El cr�dito ECTS no tiene prefijado un equivalente en horas de trabajo. Se asignan 60 cr�ditos ECTS a la carga de trabajo necesaria para superar una a�o acad�mico. Ello implica que, en funci�n de las horas lectivas
    
    
    

  \section{Motivaci�n}

    \subsection{Ontolog�as}
      Una ontolog�a no es un vocabulario ni un diccionario donde figuran las
      definiciones de los conceptos utilizados. Una ontolog�a es un mapa donde
      conceptos y significados se entrecruzan. Se trata de una forma de representaci�n
      del conocimiento que permite tener un entendimiento com�n y compartido de un
      dominio, de modo que diferentes personas o sistemas puedan compartir una misma
      visi�n de ese dominio. 

      Existen varias definiciones formales de Ontolog�a. Varios autores refieren la
      definici�n tal y como a la facilita Tom Gruber\cite{GRUBER:93}. Seg�n esa
      definici�n, una ontolog�a es una especificaci�n de una conceptualizaci�n. Otra
      definici�n m�s concreta es la ofrecida por Weigand\cite{WEIGAND:97}, seg�n el
      cual una ontolog�a es una base de datos que describe los conceptos del mundo o
      alg�n subdominio, algunas de sus propiedades, y como se relacionan cada uno de
      los conceptos. Para un sistema basado en el conocimiento, podemos asumir que
      s�lo existe aquello que podemos representar, y que todo aquello que no pertenece
      al dominio de la ontolog�a, no existe. 

      El uso de ontolog�as implica por tanto la definici�n de un vocabulario y reglas
      gramaticales que relacionen los vocablos. Estas reglas gramaticales nos
      permitir�n realizar preguntas a la ontolog�a cuyas respuestas deber�n ser,
      forzosamente, coherentes con las definiciones y constantes de la ontolog�a.
      Todas estas propiedades de las ontolog�as nos permitir�n:
      \begin{itemize}
	\item Intercambar datos entre diferentes sistemas.
	\item Crear servicios de consulta.
	\item Crear bases de conocimiento reusables.
	\item Ofrecer servicios para facilitar la interoperabilidad entre diversos
	sistemas y bases de datos.
      \end{itemize}
      
      Todas estas propiedades se pueden resumir diciendo que el uso de ontolog�as nos
      permitir� especificar una representaci�n del modelo de datos a un nivel superior
      al del dise�o de bases de datos espec�ficas, lo que permitir� la exportaci�n,
      traducci�n, consulta y unificaci�n de la informaci�n a trav�s de sistemas y
      servicios desarrollados de manera independiente.

    \subsection{Unificar conocimiento en un marco �nico y formal}
      El uso de ontolog�as para la especificaci�n del conocimiento nos permitir�
      trabajar con una �nica represetnaci�n del conocimiento, v�lida para todos los
      actores involucrados en el proceso de educaci�n. Esta visi�n �nica y global nos
      permitir� que quienes quieran que sean las personas que precisen trabajar sobre
      el conocimiento representado compartir�n todas ellas una misma visi�n de
      conjunto.
      De manera adicional, se logra que los conceptos utilizados en la descripci�n del
      conocimiento est�n definidos de manera precisa en la propia ontolog�a, con lo
      que se elimina la necesidad de acudir a fuentes externas, y por tanto se unifica
      el significado de los conceptos introducidos.
      De este modo, al unificar la representaci�n del conocimiento y las significaci�n
      de sus conceptos, logramos crear un marco �nico y formal, que posibilita:
      \begin{itemize}
	\item la eliminaci�n de ambig�edades, al trabajar todos los agentes sobre un
	mismo marco de representaci�n.
	\item un r�pido intercambio de ideas, ya que se elimina la necesidad de
	"traducci�n".
	\item dotar al modelo de conocimiento de una elevada adaptabilidad a cambios
	futuros, pues las bases han sido creadas sin ambig�edades ni incertidumbres.
      \end{itemize}

    \subsection{Marco unificado}
      %{Marco unificado para a�adir (open world assumption OWA) y extraer informaci�n.}
      Aqui tengo que hablar con �ngel, porque no s� muy bien qu� poner aqui. Busca en
      internet.

    \subsection{Hip�tesis: herramientas autom�ticas}
      Como ya hemos hablado antes, la utilizaci�n de un marco unificado para la
      representaci�n del conocimiento, nos permite homogeneizar el marco del
      conocimiento del caso de estudio, de modo que sistema y usuarios comparten la
      misma visi�n de ese mundo modelizado.
      Esta unficaci�n del marco de conocimiento nos es muy �til a la hora de utilizar
      herramientas que, de forma autom�tica, realicen las m�s variadas tareas sobre el
      marco creado, como pueden ser la extracci�n de informaci�n o la inferencia.
      Este marco �nico se ha modelado utilizando la aplicaci�n
      Proteg�\cite{Stanford-Protege:WEB}. 
      Proteg� es un editor de ontolog�as libre, con una filosof�a "`open source"' que
      adem�s soporta varios formatos, entre los que destaca OWL\cite{W3C-OWL:WEB}.
      Proteg� ha sido desarrollado por el centro de investigaci�n inform�tica y
      biom�dica\cite{Stanford-BMIR:WEB} de la Escuela Universitaria de Medicina de la
      universidad de  Standford\cite{Stanford-MED:WEB}, con la colaboraci�n de DARPA,
      eBay, National Cancer Institute, National Institute of Standards and Technology,
      National Library of Medicine y National Science Foundation, entre otros.
      Gracias al formato OWL soportado por Proteg�, a su filosof�a de software libre,
      y a los apoyos recibidos por parte de diversas instituciones como las arriba
      mencionadas, existen multitud de herramientas disponibles para trabajar de
      manera autom�tica con las ontolog�as creadas por Proteg�, ya sea en forma de
      plug-in's o en forma de herramientas que trabajan directamente sobre la
      ontolog�a en formato OWL.
      Entre la legi�n de herramientas encontradas, destacamos:
      \begin{itemize}
	\item OWL-Lint\cite{Stanford-OWL_Lint:WEB}. Herramienta para el test autom�tico
	de ontolog�as para depuraci�n, control de calidad, etc.
	\item Outline and Existential Tree Views\cite{Stanford-OETV:WEB}. Herramienta
	que nos permite la navegaci�n a trav�s de una ontolog�a, no s�lo mediante el uso
	de la relaci�n existencial es-un.
	\item Cloud Views \cite{Stanford-CV:WEB}. Herramienta que permite visualizar la
	ontolog�a como una nube de tags, cada uno con su correspondiente peso basado en
	distintos criterios (profundidad, uso, ...).
	\item Ontology browser\cite{UniManchester-OB:WEB}. Herramienta que permite la
	navegaci�n a trav�s de la ontolog�a, y que construye los documentos html de
	manera din�mica.
	\item OWLDoc\cite{Stanford-OWLDoc:WEB}. Herramienta que genera un conjunto de
	p�ginas html (la mayor�a est�ticas) para su publicaci�n en web.
	\item OWL2UML\cite{Stanford-OWL2UML:WEB}. Herramienta que crea un diagrama UML
	que representa la ontolog�a activa.
      \end{itemize}

    \subsection{Cr�tica al sistema de hojas de c�lculo disgregadas (disgregaci�n de
    la informaci�n, redundancia, incoherencia, etc.)}
      Con la llegada del nuevo plan de estudios, fue necesario redistribuir las horas
      docentes de cada asignatura (Cr�ditos ECTS) en funci�n del n�mero de horas que
      el alumno deb�a dedicar a ella para adquirir las competencias definidas por el
      plan de estudios. Con la finalidad de facilitar esta transci�n hacia una visi�n
      "`alumno"' del grado, se crearon unas hojas de c�lculo destinadas a aglutinar
      toda la informaci�n relativa a las asignaturas de un departamento y a distribuir
      las horas lectivas de cada asignatura. Este archivo de excel, consta de 8 hojas
      de c�lculo, las siguientes:
      \begin{itemize}
	\item Hoja 1: Consta de una peque�a nota con la definici�n de los diferentes
	m�todos docentes.
	\item Hoja 2: Consta de una peque�a rese�a con la definici�n de las diferentes
	actividades formativas
	\item Hoja 3: En esta hoja, denominada "`Plantilla Alumnos"' se recogen
	diferentes estad�sticas sobre los alumnos, los a�os que tardan en finalizar los
	estudios y previsiones sobre el rendimiento de los alumnos.
	\item Hoja 4: En esta hoja, llamada Plantilla Prof-Dept, se dividen los grupos
	de alumnos en cuatro clases, dependiendo del n�mero de alumnos en cada grupo.
	Luego, en funci�n del n�mero de profesores presente en el departamento y su
	disponibilidad, la hoja calcula el total de horas disponibles para la docencia.
	\item Hoja 5: En la quinta hoja del libro de excel, llamada "`Plantilla ACT y
	MET"', se recoge, en una misma hoja de c�lculo, los datos de la asignatura
	(nombre, n�mero de cr�ditos, etc), prerequisitos, n�mero de horas de cada
	actividad formativa, m�todos docentes aplicados, n�mero de horas dedicadas a la
	preparaci�n de la evaluaci�n de cada m�todos evaluador, y capacidades adquiridas
	por el alumno tras cursar esta asignatura. Al final, la hoja recoge el total de
	horas utilizadas por el alumno y las transforma en cr�ditos ECTS.
	Adicionalmente, debajo de este cuadro, aparecen repartidas las horas de docencia
	y las horas destinadas a evaluaci�n por el departamento.
	\item Hoja 6: Angel, puede ser esto otra versi�n de la hoja anterior??
	\item Hoja 7: En esta s�ptima hoja, llamada "`Plantilla C\_ESPECIFICAS"' vienen
	recogidas todas las competencias espec�ficas del plan de estudios, el n�mero de
	horas dedicadas a cada actividad formativa, las actividades formativas que
	permiten al alumno adquirir cada capacidad y el nivel adquirido por el alumno a
	la finalizaci�n de la asignatura.
	\item Hoja 8: En esta �ltima hoja del fichero, "`Resultado C\_Generales"', se
	recoge en una tabla todas las competencias generales que se pueden adquirir en
	el curso de la titulaci�n, junto con el n�mero de horas dedicadas a cada
	actividad formativa para la adquisici�n de cada competencia. Una �ltima columna
	nos indicar� el total de horas dedicadas a la adquisici�n de cada competencia.
      \end{itemize}

      Como primer apunte al m�todo empleado, podr�amos subrayar el uso de una
      herramienta como es la hoja de c�lculo para un f�n que no es el propio. Como
      consecuencia, tenemos tablas ineficientes, con muchos datos, campos con texto
      mezclados con n�meros, y que a primera vista resultan muy poco claras. Este
      sistema de hojas de c�lculo es claramente ineficiente, y su principal problema
      es la ausencia de un criterio definido a la hora de definir conceptos. Toda la
      informaci�n queda embarullada y mezclada, y resulta muy dif�cil rellenar las
      hojas para una �nica asignatura, de modo que rellenar los datos completos de
      toda una materia o incluso un grado resulta una tarea excesivamente dificultosa.
      (�SE PODR�AN ADJUNTAR LAS HOJAS DE C�LCULO A MODO DE MUESTRA?)

      En resumen, el sistema de hojas de c�lculo empleado adolece de:
      \begin{itemize}
	\item Falta de precisi�n en los conceptos.
	\item Inexistencia de l�mites en la asignaci�n de horas de trabajo a las
	asignaturas, quedando supeditada la correcci�n de la asignaci�n de horas al buen
	hacer de la persona que rellena hoja.
	\item Falta de control en la adquisici�n de competencias, quedando de nuevo
	supeditado al buen hacer de la persona que rellene las hojas de c�lculo.
	\item Exceso de informaci�n en cada hoja de c�lculo. Por ejemplo, en las
	competencias espec�ficas y generales se muestran las de todo el plan de
	estudios, en lugar de tan solo las competencias que deban ser adquiridas al
	cursar dicha asignatura.
	\item  Inexistencia de realaci�n entras las asignaturas y la materia en que se
	engloban.
	\item Invisibilidad del resto del plan. La asignatura cursada forma parte de una
	materia y esa materia de un plan de estudios. Esa relaci�n debe estar plasmada,
	dado que no son conceptos aislados, si no que est�n muy estrechamente ligadas.
      \end{itemize}

      Adem�s de estas carencias, poniendo la vista en un medio-largo plazo, resulta a
      priori muy complicado la automatizaci�n de tareas como puedan ser la asignaci�n
      de horas a cada departamento o profesor, el cambio en alguna competencia del
      plan de estudios, o la inclusi�n de alguna materia o asignatura. Es necesario
      construir las bases de manera que en un futuro la ampliaci�n del sistema o la
      modificaci�n del universo de trabajo resulte c�moda y eficiente.
      
  \section{Objetivos}
    Los objetivos principales del presente documento son:
    \begin{itemize}
      \item Crear un marco �nico y formal de conocimiento de modo que todos los conceptos
      sean inequ�vocos y cualquier persona, a�n a pesar de no estar familiarizada con
      la estructura de un plan de estudios pueda consultar, a�adir y modificar
      informaci�n al sistema. Este marco �nico garantizar� que toda la informaci�n
      incluida en el sistema sea coherente con el conocimiento modelado.
      \item Mejora de la presentaci�n de la informaci�n, de modo que sea s�lo sea visible
      aquella que tenga significado en el contexto de trabajo.
      \item Relacionar las asignaturas, con las materias, el plan de estudios y la
      normativa que rige la educaci�n superior, de modo que tanto desde un punto de
      vista mas generalista o m�s espec�fico, el usuario conozca d�nde est�
      trabajando, y c�mo se relaciona la informaci�n visualizada con el resto del
      universo modelado.
    \end{itemize}

    Con estos objetivos cumplidos, estaremos sentando las bases para tener un
    sistema estable y acotado, de modo que podamos:
    \begin{itemize}
      \item Ampliar el modelo de conocimiento hacia otroas planes de estudios, incluidos
      aquellos cursados en universidades extranjeras adheridas al marco europeo de
      educaci�n superior.
      \item Crear o aplicar herramientas autom�ticas al conocimiento modelado, que
      faciliten el trabajo con la informaci�n contenida.
    \end{itemize}

    Adem�s, con toda la documentaci�n generada, ser� una tarea sencilla la
    modificaci�n del marco de conocimiento para adpatarlo a nuevas leyes o
    nomrativas europeas para que si �stas cambian en un futuro, las herramientas
    desarrolladas y la informaci�n contenida sea f�cilmente portable hacia el nuevo
    marco educativo.

%%% Local Variables: 
%%% mode: latex
%%% TeX-master: "tfc-ontologia-grado"
%%% TeX-PDF-mode: t
%%% ispell-local-dictionary: "castellano"
%%% End: 

\cleardoublepage
\chapter{Ontolog�a de nuestro plan de grado}

\section{Herramienta utilizada:Prot�g�}
Prot�g�\cite{Stanford-Protege:WEB} es un framework para la edici�n de ontolog�as de c�digo abierto. Las ontolog�as creadas con prot�g� se pueden exportar a muy diversos formatos, entre los cuales se incluyen RDF(S), OWL e incluso esquemas XML.
Prot�g� est� construido en Java y admite extensiones creadas por los usuarios por lo que constituye una excelente base para el desarrollo de aplicaciones o prototipos.
Adicionalmente, Prot�g� cuenta con amplica comunidad de usuarios (desarrolladores, docentes, estudiantes e instituciones gubernamentales y corporaciones privadas) que utilizan Prot�g� como base de conocimiento en �reas tan diversas como la biomedicina o la adquisicion de conocimiento.

\section{Introducci�n al documento de trabajo}
Para crear la ontolog�a, se ha partido del documento \cite{FIUPM-MemoAneca}. Este documento es el remitido a ANECA\footnote{Agencia Nacional de Evaluaci�n de Calidad y Acreditaci�n - www.aneca.es} para su validaci�n de acuerdo a las normas especificadas en el marco europeo. Aneca, como miembro de ENQA\footnote{European Association for Quality Assurance in Higher Education - www.enqa.eu}, EQAR\footnote{Eurpean Quality Assurance Register for Higher Education - www.eqar.eu} y INQAAHE\footnote{International Network for Quality Assurance Agencies in Higher Education - www.inqaahe.org} es el encargado de validar los planes de estudios y de certificar el cumplimiento de los requisitos marcados por Bolonia.

Con el prop�sito de poder comprender mejor la construcci�n de la ontolog�a, vamos a comenzar haciendo un peque�o resumen de los puntos 3 y 5, referidos a los objetivos del t�tulo y a la planificaci�n de las ense�anzas.

\begin{itemize}

 \item Objetivos del t�tulo. Los objetivos generales del t�tulo son la adquisici�n por parte del egresado, de unos n�veles m�nimos de adquisici�n de una serie de capacidades, competencias y destrezas generales. Los objetivos van a fijar las capacidades m�nimas de todo alumno al finalizar los estudios.
 
 \item Competencias generales y espec�ficas. Estas capacidades, objetivos en la denominaci�n dada por el documento, son adquiridas por el alumno mediante la adquisici�n de diversas compentencias, en distintos grados. La adquisici�n programada de estas competencias es lo que permitir� al alumno cumplir con los objetivos establecidos para el plan de estudios, y poder afrontar las diversas situaciones que en su vida laboral deber� afrontar. Las competencias se diferencian entre ellas seg�n sean espec�ficas de un t�tulo (aquellas que definen procedimientos y actitudes propias de una t�tulo) o generales si lo que definen son competenciase car�cter general, aplicables a varios t�tulos. 
 En cuanto al nivel de adquisici�n de cada comeptencia, se asumir� que el alumno alcanza un nivel de adquisici�n �ptimo, de acuerdo a los objetivos propuestos, conforme va cursando las distintas asignaturas que le van otorgando esas competencias. Es tarea del evaluador, comprobar que el grado de adquisici�n de competencias por parte del alumno es el adecuado.

 \item Materias. El curso de las diferentes materias, debe asegurar la adquisici�n de todas las competencias, tanto espec�ficas como generales, definidas en el perfil del t�tulo. El trabajo fin de carrera ser� considerado como una materia m�s. La materia optatividad, no consta de ninguna asignatura, para facilitar una r�pida reacci�n ante cualquier cambio tecnol�gico, profesional o formativo que se produzca.
 
 \item Asignaturas. Debido a limitaciones del lenguaje, no se tendr�n en cuenta limitaciones sobre la extensi�n de las asignaturas en cr�ditos, limitaciones sobre la multiplicidad de los cr�ditos, ni se hablar� acerca de las horas reales a que equivale cada cr�dito ECTS. 
 Tampoco se controlar� el n�mero de asignaturas programadas por semestre, ni se determinar� qu� n�mero de cr�ditos estan destinados a la adquisici�n de competencias transversales. Tampoco se controlar� la correcta distribuci�n de cr�ditos por semestres, ni por asignaturas b�sicas, obligatorias u optativas. 
 Las asignaturas optativas quedan, por el momento, fuera del alcance de este trabajo, por no estar definidas en el documento remitido a ANECA. Se ver� con m�s detalle m�s adelante cuando se traten los individuos de que consta la ontolog�a.
 Tal y como se especifica en el documento, la inclusi�n de asignaturas como requisitos de terceras, no especifica una obligatoriedad al uso, y no limita el curso de unas asignaturas antes que otras, sino que es meramente una recomendaci�n del itinerario curricular que deber�a seguir el alumno.
 \end{itemize}
  
  
Como ya se coment� al describir el lenguaje OWL, el prop�sito de la ontolog�a es lograr una descripci�n �nica del mundo que se est� modelando. Haciendo una analog�a con la linguistica, podr�amos decir que las ontolog�as se utilizan para comprobar la sintaxis, pero no es v�lida para comprobar la sem�ntica del lenguaje. Por tanto, es posible introducir restricciones del tipo `toda asignatura debe pertenecer a una materia`, o `las competencias satisfacen los objetivos estavlecidos sobre el plan de estudios`, pero no es posible decir, en OWL `la suma de los cr�ditos de una materia debe ser igual a la suma de los cr�ditos de todas las asignaturas que lo componen`, o `las asignaturas b�sicas deben sumar 60 cr�ditos`.

\section{Explicaci�n de la ontolog�a}

Las ontolog�as OWL e proteg� est�n compuestas por clases, propiedades e individuos. Las ontolog�as OWL incluyen adem�s operadores, como uni�n, intersecci�n o negaci�n, lo que nos permite, adem�s de describir conceptos, definirlos. De este modo, los conceptos m�s complejos pueden construirse sobre definiciones de otros m�s simples, lo que facilita la concepci�n y mantenimiento de estos sistemas. 
Adem�s, el uso de ontolog�as OWL-DL, nos permite el uso de herramientas autom�ticas para comprobar que todas las sentencias y definiciones que conforman el lenguaje son consistentes, adem�s de poder razonar qu� individuos encajan en qu� descripciones, o si la jerarqu�a establecida es consistente con los individuos presentes y sus descriupciones.
Primeramente se mostrar�n las clases, pero no se hablar� de sus relaciones ni de los individuos que la componen. Posteriormente se tratar�n la propiedades de los objetos (sus relaciones), y por �ltimo se entrar� a detalle con los individuos que componen la ontolog�a, explicando las relaciones, clases y propiedades de ellos.

\begin{itemize}

  \item Clases de la ontolog�a.
  Las clases se pueden definir en proteg� como conjuntos que contienen individuos. Las clases se describen utilizando descripciones formales que definen inequ�vocamente los requisitos de pertenencia a una clase. 
  Las clases se organizan en conjuntos de superclases y subclases, que forma la taxonom�a de nuestro universo en observaci�n. Esta taxonom�a puede ser obtenida de manera autom�tica por un razonador, que tambi�n puede comprobar su consistencia.
  Se han definido siete clases para la creaci�n de nuestra ontolog�a.	
    \begin{enumerate}
      \item Objetivo\_General.
      Con la clase Objetivo\_General queremos definir el conjunto de individuos que componen los objetivos generales del t�tulo. Esta clase no cuenta con ninguna definici�n, ya que los objetivos generales vienen definidos en el t�tulo. El �nico criterio de pertenencia posible a esta clase, es la definici�n de objetivos en el plan de estudios.
      \item Competencia.
      La clase Competencia agrupa a todas las competencias que es preciso adquirir para lograr el cumplimiento de los objetivos generales establecidos. Se definen como compentencias todos aquellos individuos que representan aptitudes y actitudes, cuya adquisici�n por parte del alumno conlleva la consecuci�n de los objetivos generales propuestos para la obtenci�n del t�tulo. Formalmente, se define como aquellos individuos que est�n relacionados con individuos de la clase Objetivo\_General mediante la propiedad que defina la adquisici�n, al menos una vez y que s�lo tengan relaci�n con individuos de la clase Objetivo\_General. M�s adelante veremos la sintaxis de esta descripci�n en proteg�, y definiremos la propiedad formalmente. Dentro de esta clase, podemos definir dos grupos de compentecias, seg�n sean competencias espec�ficas (propias de la rama del conocimiento donde se enmarca la titulaci�n) o generales (comunes a todas las ramas del conocimiento y que suelen referirse a aptitudes y actitudes m�s que a 
conocimientos o metodolog�as).
      \begin{description}
	\item Competencia\_General. La clase Competencia\_General engloba todas las competencias que es preciso adquirir para el cumplimiento de los objetivos generales del t�tulo, y que son comunes a otras ramas del conocimiento, como por ejemplo ``Saber trabajar en situaciones carentes de informaci�n y bajo presi�n, teniendo nuevas ideas y siendo creativo''. Se trata de una competencia que bien podr�a servir para cualquier otra titulaci�n, no �nicamente para la que estamos modelando. En general, est�n orientadas a la adqusici�n de aptitudes, actitudes y capacidades enfocadas hacia aspectos humanos, (como el trabajo en equipo, la motivaci�n, o la actualizaci�n de conocimientos de manera aut�noma, por citar algunos), que ser�n �tiles al alumno en en su incorporaci�n al mundo laboral. Son competencias generales todos aquellos individuos que son competencias y que adem�s son otorgadas de forma espec�fica al cursar una materia.
	\item Competencia\_Especifica. La clase Competencia\_Especifica agrupa aquellas competencias que es preciso adquirir para el cumplimiento de los objetivos generales del t�tulo, y que son espec�ficos de la rama del conocimiento propia de la titulaci�n, como por ejemplo ``Concebir y desarrollar sistemas digitles utilizando lenguajes de descripci�n hardware''. Son competencias muy espec�ficas de la titulaci�n, y que fuera de ese �mbito su aplicaci�n ser�a poco o nada �til. Suelen estar orientadas m�s hacia aspectos pr�cticos de la titulaci�n e incluyen la adquisici�n de metodolog�as y rutinas de trabajo y desarrollo eminentemente pr�cticas. Se consideran competecias espec�ficas todos aquellos individuos que son compentecias y que adem�s son otorgadas de forma espec�fica al cursar una materia
      \end{description}
      Como hemos visto, compentencias generales y espec�ficas son distintas entre s�, pero los individuos de ambas clases adquieren su condici�n de competencia espec�fica/general cuando son adquiridas al cursar una materia, sin disntinci�n entre unas y otras. Por cuestiones de dise�o, y para facilitar la comprensi�n del modelo se ha optado por separar en dos clases las competencias espec�ficas y generales. �C�mo distinguir� proteg� entre las competencias espec�ficas y generales? Los individuos pertenecientes a cada una de las clases, se relacionar�n con las materias mediante dos relaciones distintas, creadas ad hoc para cada una de las clases. De este modo logramos que el razonador, una vez haya clasificado un individuo como perteneciente a la clase competencia, pueda clasificarlo como general o espec�fica en funci�n de la propiedad que lo relacione con las materias, ganando en potencia de c�lculo, auqnue para ello la ontolog�a deba de ser algo m�s compleja.
      \item Materia.
      La clase Materia aglutina todos los individuos que representan las diferentes materias de que se compone la titulaci�n. Coloquialmente podemos entender las materias como aquellos individuos que permiten al alumno adquirir competencias, de modo que puedan cumplir con las objetivos generales del t�tulo. Formalmente hablando, son materias todos aquellos individuos relacionados con alg�n individuo de la clase Competencia, y al menos una vez con alguna Competencia\_General. Posteriormente se definir� la propiedad que une Materia con Compentencia y con Competencia\_General. 
      \item Asignatura.
      En la clase Asignatura se agrupan todos los individuos que representan las diferentes asignaturas de que se compone cada materia. Las definiciones de asignatura y de materia son complementarias: Podemos definir una materia como el conjunto de asignaturas de un mismo �mbito, o bien podemos definir una asignatura como parte integrante de una materia, con quien comparte la naturaleza de los conocimientos contenidos. Dado el car�cter m�s general de la materia, se ha optado en la ontolog�a por que sean las materias a partir de las cuales se definan las asignaturas, y no al contrario. Ademas, esa decisi�n simplifica el modelo, ya que como veremos m�s adelante, tambi�n actividades formativas y m�todos docentes depender�n para su definici�n en el modelo de la clase Materia. Formalmente descrito, podemos definir una asignatura como todo aquel individuo relacionado con alg�n individuo de la clase Materia al menos una vez y s�lo con Materias, de modo que la asignatura forme parte de dicha Materia.
      \item Actividad\_Formativa.
      La clase Actividad\_Formativa define todos aquellos individuos que tienen su correspondencia en el mundo real con las distintas actividades formativas que pueden desarrollarse para el aprendizaje de la asignatura. Una actividad formativa es la actividad a realizar por el profesor y el alumnado a lo largo de un curso, diferenci�ndose unas de otras en el prop�sito buscado por la acci�n did�ctica. Por tanto, para una misma materia, concurren variadas actividades formativas, ponder�ndose su distribuci�n a lo largo del curso de dicha materia en funci�n de los objetivos propuestos en el plan de estudios.
      Por tanto, de cara a nuestra ontolog�a, definiremos la clase actividad formativa como el conjunto de individuos que representa actividades did�cticas, que se utilizan  para impartir una o varias materias. En t�rminos formales, ser�an todos los individuos que est�n relacionados al menos una vez con al menos un individuo de clase materia. Como en los casos ante3riores, cuando definamos las propiedades de cada objeto, veremos estas definiciones plasmadas en sintaxis de proteg�.
      \item Metodo\_Docente.
      Un m�todo docente es un conjunto de formas, procedimientos, t�cnicas, actividades, etc, de ense�anza y aprendizaje. Para una misma materia, es por tanto comatible el uso de diversos m�todos docentes compaginados con las distintas actividades formativas. Algunas actividades formativas y algunos m�todos docentes ser�n mutuamente excluyentes, pero por regla general pueden conbinarse entre ellos a criterio del docente. 
      En la ontolog�a definimos un m�todo docente como el conjunto de procedimientos y t�cnicas de ense�anza y aprendizaje utilizados para impartir una materia. M�s formalmente, podr�amos definir un m�todo docente como aquellos individuos que est�n relacionados al menos una vez con al menos un individuo de la clase materia. M�s tarde veremos la definici�n de las propiedades en sintaxis de proteg�.
    \end{enumerate}

  \item Propiedades de la ontolog�a.
  Las propiedades son relaciones binarias entre individuos, es decir, una propiedad une dos individuos entre s�. Las propiedades pueden tener inversas, pueden ser funcionales, transitivas, sim�tricas... Estas relaciones se pueden dar tanto entre individuos de la misma clase, como entre individuos de distintas clases...
  Un razonador autom�tico puede computar si una relaci�n entre dos individuos es consistente con el resto de la ontolog�a. 
  Es de destacar que las propiedades �nicamente se pueden establecer entre dos individuos. No existen propiedades con cardinalidad tres, lo que implicar� que en el caso de que que sea preciso establecer una relaci�n a tres, ser�a preciso modelarlo como la relaci�n binaria entre el producto escalar de dos de ellas sobre la tercera. 
  
  \begin{enumerate}
    \item OG\_seCumpleMedianteLaAdquisicionDe\_CO.
    Dominio: 	Objetivo\_General.
    Rango: 	Competencia.
    Inversa: 	CO\_seAdquiereParaCumplir\_OG
    Se trata de la relaci�n que existe entre individuos pertenecientes a la clase Objetivo\_General y Competencia, mediante la cual se modela el hecho de que los objetivos generales de la titulaci�n se cumplen al adquirirse ciertas competencias definidas en la gu�a de la titulaci�n. 
    \item CO\_seAdquiereParaCumplir\_OG.
    Dominio: 	Competencia.
    Rango: 	Objetivo\_General.
    Inversa: 	OG\_seCumpleMedianteLaAdquisicionDe\_CO.
    Las competencias, ya sean estas espec�ficas o generales, se adquieren con el fin de poder cumplir con los objetivos generales establecidos en la gu�a de la titulaci�n. 
    \item CG\_esOtorgadaPor\_MA.
    Dominio: 	Competencia\_General.
    Rango: 	Materia.
    Inversa: 	MA\_otorgaCompetenciasGenerales\_CG.
    Un alumno adquiere una cierta competencia general al cursar la materia que la otorga. Se podr�a decir de otro modo: Para que un alumno adquiera una competencia general, es preciso que curse una cierta materia.
    \item CE\_esOtorgadaPor\_MA.
    Dominio: 	Competencia\_Especifica.
    Rango: 	Materia.
    Inversa: 	MA\_otorgaCompetenciasEspec�ficas\_CE.
    Un alumno adquiere una cierta competencia espec�fica al cursar la materia que la otorga. Se podr�a decir de otro modo: Para que un alumno adquiera una competencia espec�fica, es preciso que curse una cierta materia.
    \item MA\_otorgaCompetenciasGenerales\_CG.
    Dominio: 	Materia.
    Rango: 	Compentencia\_General.
    Inversa: 	CG\_esOtortgadaPor\_MA.
    El curso de una materia por parte de un alumno, le otorga ciertas competencias generales definidas en la ficha de cada materia. 
    \item MA\_otorgaCompetenciasEspec�ficas\_CE.
    Dominio: 	Materia.
    Rango: 	Compentencia\_Espec�fica.
    Inversa: 	CE\_esOtortgadaPor\_MA.
    El curso de una materia por parte de un alumno, le otorga ciertas competencias espec�ficas definidas en la ficha de cada materia.
    \item MA\_constaDe\_AS.
    Dominio: 	Materia.
    Rango: 	Asignatura.
    Inversa: 	AS\_formaParteDe\_MA.
    Una materia est� compuesta por varias asignaturas, que quedan de ese modo agrupadas bajo esa materia. Se podr�a decird de otro modo: que las asignaturas quedan agrupadas en diversas materias. Es una relaci�n inversamente funcional, es decir, su funci�n inversa es funcional. Dicho coloquialmente, significa que una materia puede constar de una o varias asignaturas, pero una asignatura s�lo puede pertenecer a una materia.
    \item MA\_seImparteMediante\_ME.
    Dominio: 	Materia.
    Rango: 	Metodo\_Docente.
    Inversa: 	ME\_utilizadoParaImpartir\_MA.
    Las materias se imparten seg�n dictan ciertos m�todos docentes considerados adecuados por los docentes de la titulaci�n. Estos m�todos docentes pueden ser complementarios entre s�, y en ning�n caso son excluyeentes entre ellos.
    \item MA\_seImparteSeg�n\_AF.
    Dominio: 	Materia.
    Rango: 	Actividad\_Formativa.
    Inversa: 	AF\_utilizadaParaImpartir\_MA.
    Las materias se imparten realizando ciertas activides formativas definidas por los docentes de la materia. Los cr�ditos de docencia de esa materia, deben estar por tanto distribuidos entre las distintas actividades formativas. Al igual que ocurre con los m�todos docentes, las diferentes actividades formativas no son excluyentes entre s�, sino complementarios.
    \item AS\_esRequisitoPara\_AS.
    Dominio: 	Asignatura.
    Rango: 	Asignatura.
    Inversa: 	AS\_tieneComoRequisito\_AS.
    Como ya se explic� antes, el hecho de que una asignatura tenga como requisito en la gu�a de la titulaci�n el haber cursado una asignatura anteriormente, no implica que obligatoriamente se haya de cursar esa asignatura con anterioridad. Es decir, se trata de una mera recomendaci�n de cara al itinerio a seguir en la titulaci�n, y en ning�un caso de obligado cumplimiento. Se ha tenido en cuenta a la hora de elaborar la ontolog�a, ya que el f�n �ltimo de esta es proporcionar una herramienta para un dise�o de la titulaci�n correcto, lo que incluye el trayecto curricular del alumno, y no para el control del desarrollo del alumno. Por tanto, dado que lo que estamos haciendo es ayudar a dise�ar el t�tulo, vamos a incluir los requisitos para el curso de asignaturas, como si de cumplimiento obligatorio se tratase.
    Es una relaci�n transitiva, es decir, que si la asignatura ``A'' es requisito de la asignatura ``B'' y a su vez ``B'' es requisito de la asignatura ``C'', entonces la asignatura ``A'' es un requisito de la asignatura ``C''.
    \item AS\_formaParteDe\_MA.
    Dominio: 	Asignatura.
    Rango: 	Materia.
    Inversa: 	MA\_constaDe\_AS.
    Las asignaturas se agrupan por tem�tica en diversas materias, de las que forman parte. Se trata de una relaci�n funcional, dado que una asignatura s�lo puede pertenecer a una materia, mientras que una materia puede estar compuesta por varias asignaturas.
    \item AS\_tieneComoRequisito\_AS.
    Dominio: 	Asignatura.
    Rango: 	Asignatura.
    Inversa: 	AS\_esRequisitoPara\_AS.
    Es la relaci�n inversa de AS\_esRequisitoPara\_AS, y al tratarse con rango y dominio coincidentes, su explicaci�n y detalle es el mismo que el explicado para su inversa. 
    \item ME\_utilizadaParaImpartir\_MA.
    Dominio: 	Metodo\_Docente.
    Rango: 	Materia.
    Inversa: 	MA\_seImparteMediante\_ME.
    Los m�todos docentes se siguen al impartir las diferentes materias, seg�n lo establecido en la gu�a del t�tulo, de modo que la adquisici�n de competencias por parte del alumno pueda ser �ptima. 
    \item AF\_utilizadaParaImpartir\_MA.
    Dominio: 	Actividad\_Formativa.
    Rango: 	Materia.
    Inversa: 	MA\_seImparteSeg�n\_AF.
    Las actividades formativas se realizan para impartir las diferentes materias, seg�n cierto n�mero de cr�ditos especificado en la gu�a del plan de estudios. Adem�s se ha considerado importante, de cara al dise�o del t�tulo, conocer cu�ntos cr�ditos se dedican a cada actividad formativa. M�s adelante, en la descripci�n de los distintos individuos, veremos c�mo se almacena esa informaci�n. 
    
  \end{enumerate}

\item Individuos de la ontolog�a.
  Los individuos representan objetos de la ontolog�a en el dominio que estamos estudiando. Proteg� no hace uso del Unique Name Assumption, es decir, para proteg� dos individuos pueden referirse al mismo objeto del mundo real, salvo que se especifique lo contrario. Esta es una consecuencia de las ontolog�as OWL: todo lo que no sea dicho de forma expl�cita puede ser cierto. El hecho de que no especifiquemos si dos individuos son o no los mismos, significa que pueden o no serlo, para ese dominio. En nuestra ontolog�a todos los individuos que componen la ontolog�a son distintos, lo que se especificara correctamente en la definici�n de cada individuo.
  \begin{description}
   \item OBJ01: Representa
  \end{description}


  
    
\end{itemize}


\section{Instancia UPM (varios ejemplos?)}
aplicar la instancia de nuestro plan. si es posible, incluir varios ejemplos.
\cleardoublepage
\chapter{Generalizaci�n}
hablar de las posibilidades de "portabilidad" que tiene la herramienta
\section{Intro: �qui�n puede usar esta ontolog�a? �La UCM? S�, no, pq, etc.}
hablar de a qui�n le ser�a util la ontolog�a y porqu�.
\section{Idea: jerarquia (refinamiento desde la ley hasta los planes)}
refinar el modelo hasta la ley, o explicarlo a la inversa, desde la ley.
\section{Test: Instancia UCM?}
hacer pruebas con otras instancias.
\cleardoublepage
\chapter{Aplicaciones (autom�ticas)}
Aqui hablaremos de las disitntas aplicaciones, autom�ticas o no, que hemos encontrado y c�mo pensamos que le podemos sacar partido a todo esto.

\section{Mencionar nuestras ideas sobre c�mo sacar partido a esto}

\todo{Hablar los puntos mencionados por }
\begin{itemize}
\item Se supone que la \emph{Description Logic} permite el
  razonamiento sobre la ontolog�a para as� descubrir errores en su
  concepci�n (TBox) o en sus datos (ABox).
\item Presentaci�n de la informaci�n contenida en la ontolog�a para
  diferentes perfiles: profesores, alumnos, secretar�a, legos,
  etc. V�a exportaci�n a html o a herramientas visuales (grafos,
  etc.).
\item La ontolog�a es en s� misma un modelo de datos por lo que se
  puede generar a partir de ella diferentes modelos de datos: UML,
  relacional, no-sql, etc. (Daniel tiene algunas cosas apuntadas de un
  antiguo org).
\end{itemize}

\section{Ejemplo de razonamiento (puede que encuentres una inconsistencia al ir metiendo la info en la ontolog�a,ap�ntalo).}

\section{Ejemplo de vis. de la informaci�n (html).}

%%% Local Variables: 
%%% mode: latex
%%% TeX-master: "tfc-ontologia-grado"
%%% TeX-PDF-mode: t
%%% ispell-local-dictionary: "castellano"
%%% End: 

\cleardoublepage
\chapter{Conclusiones}

%%% Local Variables: 
%%% mode: latex
%%% TeX-master: "tfc-ontologia-grado"
%%% TeX-PDF-mode: t
%%% ispell-local-dictionary: "castellano"
%%% End: 

\cleardoublepage

%%%%%%%%%%%%%%%%%%%%%%%%%%%%%%%%%%%%%%%%%%%%%%%%%%%%%%%%%%%%%%%%%%%%%%
%% COMIENZO DE LOS AP�NDICES
%% (NO TOCAR)
\appendix

%%%%%%%%%%%%%%%%%%%%%%%%%%%%%%%%%%%%%%%%%%%%%%%%%%%%%%%%%%%%%%%%%%%%%%
%% AP�NDICES:
%%  * PONER CADA AP�NDICE EN UN FICHERO COMO SI FUERAN CAP�TULOS
%%  * HACER UN include POR CADA FICHERO E INCLUIRLO EN includeonly
%%  * A�ADIR \clearemptydoublepage DESPU�S DE CADA include
\chapter{T-Box en sintaxis M�nchester}

\lstset{language=owlms,literate=,mathescape}

\lstinputlisting{t-box.ms}

%%% Local Variables: 
%%% mode: latex
%%% TeX-master: "tfc-ontologia-grado"
%%% TeX-PDF-mode: t
%%% ispell-local-dictionary: "castellano"
%%% End: 

\cleardoublepage

%%%%%%%%%%%%%%%%%%%%%%%%%%%%%%%%%%%%%%%%%%%%%%%%%%%%%%%%%%%%%%%%%%%%%%
%% COMIENZO DE LA BIBLIOGRAF�A
%% (NO TOCAR)
\bibliographystyle{plain}
\addcontentsline{toc}{chapter}{Bibliograf�a}

%%%%%%%%%%%%%%%%%%%%%%%%%%%%%%%%%%%%%%%%%%%%%%%%%%%%%%%%%%%%%%%%%%%%%%
%% BIBLIOGRAF�A
%% (A�ADIR FICHEROS .bib CON LA BIBLIOGRAF�A USADA EN EL TFC)
\bibliography{bib}
\nocite{*}

\end{document}

%%% Local Variables: 
%%% mode: latex
%%% TeX-master: t
%%% TeX-PDF-mode: t
%%% ispell-local-dictionary: "castellano"
%%% End: 
