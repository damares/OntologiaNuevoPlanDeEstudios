\subsection{Ontolog�as}
Una ontolog�a no es un vocabulario ni un diccionario donde figuran las definiciones de los conceptos utilizados. Una ontolog�a es un mapa donde conceptos y significados se entrecruzan. Se trata de una forma de representaci�n del conocimiento que permite tener un entendimiento com�n y compartido de un dominio, de modo que diferentes personas o sistemas puedan compartir una misma visi�n de ese dominio. 

Existen varias definiciones formales de Ontolog�a. Varios autores refieren la definici�n tal y como a la facilita Tom Gruber\cite{GRUBER:93}. Seg�n esa definici�n, una ontolog�a es una especificaci�n de una conceptualizaci�n. Otra definici�n m�s concreta es la ofrecida por Weigand\cite{WEIGAND:97}, seg�n el cual una ontolog�a es una base de datos que describe los conceptos del mundo o alg�n subdominio, algunas de sus propiedades, y como se relacionan cada uno de los conceptos. Para un sistema basado en el conocimiento, podemos asumir que s�lo existe aquello que podemos representar, y que todo aquello que no pertenece al dominio de la ontolog�a, no existe. 

El uso de ontolog�as implica por tanto la definici�n de un vocabulario y reglas gramaticales que relacionen los vocablos. Estas reglas gramaticales nos permitir�n realizar preguntas a la ontolog�a cuyas respuestas deber�n ser, forzosamente, coherentes con las definiciones y constantes de la ontolog�a.
Todas estas propiedades de las ontolog�as nos permitir�n:
\begin{itemize}
\item Intercambar datos entre diferentes sistemas.
\item Crear servicios de consulta.
\item Crear bases de conocimiento reusables.
\item Ofrecer servicios para facilitar la interoperabilidad entre diversos sistemas y bases de datos.
\end{itemize}
Todas estas propiedades se pueden resumir diciendo que el uso de ontolog�as nos permitir� especificar una representaci�n del modelo de datos a un nivel superior al del dise�o de bases de datos espec�ficas, lo que permitir� la exportaci�n, traducci�n, consulta y unificaci�n de la informaci�n a trav�s de sistemas y servicios desarrollados de manera independiente.

Falata buscar algo de OWL


