\subsection{Unificar conocimiento en un marco �nico y formal}
El uso de ontolog�as para la especificaci�n del conocimiento nos permitir� trabajar con una �nica represetnaci�n del conocimiento, v�lida para todos los actores involucrados en el proceso de educaci�n. Esta visi�n �nica y global nos permitir� que quienes quieran que sean las personas que precisen trabajar sobre el conocimiento representado compartir�n todas ellas una misma visi�n de conjunto.
De manera adicional, se logra que los conceptos utilizados en la descripci�n del conocimiento est�n definidos de manera precisa en la propia ontolog�a, con lo que se elimina la necesidad de acudir a fuentes externas, y por tanto se unifica el significado de los conceptos introducidos.
De este modo, al unificar la representaci�n del conocimiento y las significaci�n de sus conceptos, logramos crear un marco �nico y formal, que posibilita:
\begin{itemize}
\item la eliminaci�n de ambig�edades, al trabajar todos los agentes sobre un mismo marco de representaci�n.
\item un r�pido intercambio de ideas, ya que se elimina la necesidad de "traducci�n".
\item dotar al modelo de conocimiento de una elevada adaptabilidad a cambios futuros, pues las bases han sido creadas sin ambig�edades ni incertidumbres.
\end{itemize}



